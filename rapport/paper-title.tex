% paper-title.tex -- example of modular LaTeX
% Assad Ebrahim
% May 13, 2010
% ***********************************

% THIS DOCUMENT HOLDS BOILERPLATE FOR YOUR TITLE PAGE AND BUILDS YOUR PAPER
% 	- pulls in your authors file (see line 15)
% 	- pulls in your TeX definitions file (see line 20)
% 	- pulls in your content file(s) to fill the body (see line 42}	
%	- pulls in your bibliography (see line 46)

% title page material
\documentclass[11pt,a4paper,oneside,pdftex]{article}

\usepackage{listings}
\usepackage{textcomp}
\usepackage{hyperref}

\hypersetup{
	colorlinks,
	citecolor=black,
	filecolor=black,
	linkcolor=black,
	urlcolor=black
}

% Code listings
\lstset{ %
language=Prolog,                % the language of the code
basicstyle=\footnotesize,       % the size of the fonts that are used for the code
numbers=left,                   % where to put the line-numbers
numberstyle=\footnotesize,      % the size of the fonts that are used for the line-numbers
stepnumber=1,                   % the step between two line-numbers. If it's 1, each line 
                                % will be numbered
numbersep=5pt,                  % how far the line-numbers are from the code
showspaces=false,               % show spaces adding particular underscores
showstringspaces=false,         % underline spaces within strings
showtabs=false,                 % show tabs within strings adding particular underscores
frame=single,                   % adds a frame around the code
tabsize=2,                      % sets default tabsize to 2 spaces
captionpos=b,                   % sets the caption-position to bottom
breaklines=true,                % sets automatic line breaking
breakatwhitespace=false,        % sets if automatic breaks should only happen at whitespace
title=\lstname,                 % show the filename of files included with \lstinputlisting;
                                % also try caption instead of title
escapeinside={\%*}{*)},         % if you want to add a comment within your code
morekeywords={*,...}            % if you want to add more keywords to the set
}

\author{
	Stefan Åhman \\ 900326-2376 \\ sahman@kth.se
\and
	Marcus Wallstersson \\ 880301-6099 \\mwallst@kth.se}
\date{\today}  % static date of your choice, and an automated revision date

%%% INCLUDE macros, headers, and so on -- MODULE
% Custom LaTeX Definitions File
% Author: Assad Ebrahim, assad.ebrahim@mathscitech.org
% Created: Aug 4, 2008
% License: This file can be used freely provided the author attribution 
%          and this license information is maintained.
% ----------------------------------------------------

% PACKAGES AND ENVIRONMENTS BLOCK: LIST HERE PACKAGES WITH DESIRED FUNCTIONALITY %

\usepackage[utf8]{inputenc}

% Symbology
\usepackage{amsfonts} % for $\mathbb{R}$ reals, $\mathbb{Z}$ integers, $\mathbb{R}^n$ $n$-dim. spaces
\usepackage{amsmath}  % for \eqref
\usepackage{amssymb}  % for full AMS mathematics symbology

% Graphics
\usepackage[dvips]{graphics}       	%for embedding graphics files
\usepackage{graphicx}  				%for \includegraphics{file.eps}  % note EPS format
		% (can make a PNG into EPS using png2EPS)

% Set Margins:
%\usepackage{fullpage}  %for not so huge BOOK margins.  Gives std 1" margins all around.
		% my opinion: this is much less pleasantly readable.
% Manually setting margins
%	\addtolength{\oddsidemargin}{-.375in}
%	\addtolength{\evensidemargin}{-.375in}
%	\addtolength{\textwidth}{.75in}
%	\addtolength{\topmargin}{-.5in}
\addtolength{\textheight}{1.0in}  %reduces bottom margin -- more per page, but still good side margins

% Code Listings and formatted dynamically included program listings
\usepackage{verbatim}  	%for formatting dynamically included program listings
						% e.g. {\footnotesize\verbatiminput{c:/tmp/file.tex}} 

% Defined Code Listing Environment
% for creating environments, see http://theoval.cmp.uea.ac.uk/~nlct/latex/novices/newenvironment.html
\newenvironment{code}[1]  % takes pathfilenametocode as argument
{
{\footnotesize \verbatiminput{#1}}% begin code
{}
}    % end code

\newenvironment{proof}%[1] % takes Proof Title as (optional) argument
{
{Proof\\ 
%\begin{proof}{#1}\end{proof} 
\footnotesize}
{}
}

% MACRO DEFINITIONS %
\newtheorem{theorem}{Theorem} %[section]
\newtheorem{claim}[theorem]{Claim}
\newtheorem{proposition}[theorem]{Proposition}
\newtheorem{lemma}[theorem]{Lemma}
\newtheorem{corollary}[theorem]{Corollary}
\newtheorem{fact}[theorem]{Fact}
\newtheorem{definition}[theorem]{Definition}
\newtheorem{question}[theorem]{Question}
\newtheorem{keypoint}[theorem]{Key Point}
\newtheorem{keyconcept}[theorem]{Key Concept}
\newtheorem{problem}[theorem]{Problem}
\newtheorem{exercise}[theorem]{Exercise}
\newtheorem{algorithm}[theorem]{Algorithm}
%\newtheorem{proof}[theorem]{Proof}
%Example: 
%	\begin{lemma}\label{theorem_trivial}<add your Lemma Name>\end{lemma}
%	<add your lemma text>
%   Later, to refer to the label: Theorem \ref{theorem_trivial}.  Note run LaTeX twice to get refs correct.

% NUMBERING CONVENTIONS %
%\numberwithin{equation}{section}  % or {subsection}  % gives equation numbering by section of text
%\numberwithin{theorem}{section}   % numbering choice for theorems, etc.

% MACRO COMMANDS %already defined: \lim, \sin, \min, \arctan, etc.
\newcommand{\qed}{\ensuremath{_\square}} %for end of proof symbol
\newcommand{\lcm}{\ensuremath{\mbox{lcm} }} % LCM -- because \gcd is defined
\newcommand{\imply}{\Longrightarrow} % for =>
\newcommand{\limply}{\Longleftarrow} % for =>
\newcommand{\biimply}{\Leftrightarrow} % for <=>
\newcommand{\goesto}{\rightarrow} % for limit ->
\newcommand{\fuzzy}[1]{\ensuremath{ \underset{\sim}{#1} }} % usage: \fuzzy{A} for A under-tilde

%Macro for: \begin{equation} formula_text \end{equation}
\newcommand{\be}{\begin{equation}}
\newcommand{\ee}{\end{equation}}
%Labeling equations:  \be ... \label{labelname} \ee
%Later, to refer to the label: \ref{labelname} or \eqref{labelname}
  % include from separate file

% \pagestyle{myheadings} % use page headers
% puts headers on both pages alternating even and odd numbers.  \markboth{even-text}{odd-text}
% \markboth{Modular Examples for LaTeX, August 2008, Assad Ebrahim}{(Odd Numbered Pages), (Additional Text), (Final Text)}

\setlength{\parindent}{0.0in}
\setlength{\parskip}{0.08in}

\graphicspath{{./images/}}

\title{Laboration 2: Modellprovning för CTL}

%%%%%%% DOCUMENT BEGIN
\begin{document}
\bibliographystyle{swealpha}
%\bibliographystyle{alpha}  % BibTeX bibliography.  Available styles: plain, alpha, unsrt, abbrv
\maketitle	% makes title
\begin{center}
KTH Kista, Stockholm
\end{center}
\thispagestyle{empty}

\newpage
\renewcommand{\contentsname}{Innehållsförteckning}
\tableofcontents
\thispagestyle{empty}

\newpage  % separates table of contents from body of document

% vvvvvvvvvvvvvv  BEGIN BODY  vvvvvvvvvvvvvvvvvvv

\setcounter{page}{1}

% PULL IN AND ORDER YOUR CONTENT MODULES HERE
\section{Inledning}
För att kunna kontrollera om en temporallogisk formel \varphi gäller i ett visst tillstånd \textit{s} i en given modell M kan man använda sig av en modellprovare. Detta programverktyg måste i denna laboration implementeras att hantera följande delmängd CTL-reglerna (Computation tree logic):

formler

För att kunna kontrollera om en temporallogisk formel \varphi gäller i ett visst tillstånd \textit{s} i en given modell M kan man använda sig av en modellprovare. Detta programverktyg måste i denna laboration implementeras att hantera följande delmängd CTL-reglerna (Computation tree logic):
\section{Problem och Syfte}

\clearpage
\section{Genomförande}

\section{Frågor}
\label{sub:fragor}

\renewcommand{\labelenumi}{(\alph{enumi})}
\begin{enumerate}
\item Vad skiljer labbens version av CTL från bokens version?

Labbens implementation av CTL kan inte hantera negation av CTL-formler, “U” Until och
“implicerar” som tas upp i \cite{huth}.

\item Hur kan man utöka modellprovaren så att den hanterar bokens CTL?

För att kunna hantera negerade formler krävs det att De Morgans lagar implementeras i
modelltestaren. 

\item Hur hanterade ni variabelt antal premisser (som i AX-regeln)?

Men en hjälpfunktion “acheck” som rekursivt behandlar alla states som kan nås från det aktuella tillståndet. Kontrollerar alla dessa möjliga tillstånd med den ursprungliga funktionen “check” där alla tester måste evalueras till true.

\end{enumerate}
\section{Resultat}
\subsection{Frågor} % (fold)
\label{sub:fragor}

% subsection fr_gor (end)
\renewcommand{\labelenumi}{(\alph{enumi})}
\begin{enumerate}
\item Vad skiljer labbens version av CTL från bokens version?

Labbens implementation av CTL kan inte hantera negation av CTL-formler, “U” Until och
“implicerar” som tas upp i \cite{huth}.

\item Hur kan man utöka modellprovaren så att den hanterar bokens CTL?

För att kunna hantera negerade formler krävs det att De Morgans lagar implementeras i
modelltestaren. 

\item Hur hanterade ni variabelt antal premisser (som i AX-regeln)?

Men en hjälpfunktion “acheck” som rekursivt behandlar alla states som kan nås från det aktuella tillståndet. Kontrollerar alla dessa möjliga tillstånd med den ursprungliga funktionen “check” där alla tester måste evalueras till true.

\end{enumerate}
\section{Slutsats}

Prolog som verktyg för bevissökningar visade sig vara mycket effektivt och lättanvänt. Genom att skapa en modell med CTL-regler för ett problem går det med villkor dra paralleller mellan verkligheten och tillståndsgrafen, huruvida ens programidéer kan realiseras.

Efter att ha slutfört laborationen har ett förstående för CTL utvecklats. Att verifiera bevis för en tillståndsgraf med den implementerade beviskontrolleraren fungerade mycket bra.
\clearpage
\section{Bilagor}
Här presenteras programkoden för modellprovaren och de egenskrivna testerna.
\subsection{Programkod}
\label{sub:programkod}
\lstinputlisting{code/lab2.pl}
\clearpage
\subsection{Tester}
\label{sub:tester}
\lstinputlisting{code/valid1000.txt}
\lstinputlisting{code/invalid1001.txt}


% ^^^^^^^^^^^^^   END OF BODY   ^^^^^^^^^^^^^^^^^^^^^^^^^^
\newpage  % if you wish your References section on a separate page
\renewcommand\refname{Referenser}
\clearpage
\addcontentsline{toc}{section}{Referenser}
\bibliography{bibliog}
\end{document}
%%%%%%% DOCUMENT END
