\section{Resultat}

\subsection{Testkörningar}
\label{sub:test}

Alla tester som fanns givna för laborationen fungerade utmärkt. De egna tester som skapades för de CTL-formler som beskrev trafikljuset gav önskat resultat. Formeln \texttt{ef(ag(ex(o)))} bevisades vara sann och \texttt{and(r,ax(g))} falsk. Testerna finns bifogade i kapitel~\ref{sub:tester}.


\subsection{Frågor}
\label{sub:fragor}

\renewcommand{\labelenumi}{(\alph{enumi})}
\begin{enumerate}
\item Vad skiljer labbens version av CTL från bokens version?

Labbens implementation av CTL kan inte hantera negation av CTL-formler, “U” Until och
“implicerar” som tas upp i \cite{huth}.

\item Hur kan man utöka modellprovaren så att den hanterar bokens CTL?

För att kunna hantera negerade formler krävs det att De Morgans lagar implementeras i
modelltestaren. 

\item Hur hanterade ni variabelt antal premisser (som i AX-regeln)?

Men en hjälpfunktion “acheck” som rekursivt behandlar alla states som kan nås från det aktuella tillståndet. Kontrollerar alla dessa möjliga tillstånd med den ursprungliga funktionen “check” där alla tester måste evalueras till true.

\end{enumerate}