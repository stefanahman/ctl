% Custom LaTeX Definitions File
% Author: Assad Ebrahim, assad.ebrahim@mathscitech.org
% Created: Aug 4, 2008
% License: This file can be used freely provided the author attribution 
%          and this license information is maintained.
% ----------------------------------------------------

% PACKAGES AND ENVIRONMENTS BLOCK: LIST HERE PACKAGES WITH DESIRED FUNCTIONALITY %

\usepackage[utf8]{inputenc}

% Symbology
\usepackage{amsfonts} % for $\mathbb{R}$ reals, $\mathbb{Z}$ integers, $\mathbb{R}^n$ $n$-dim. spaces
\usepackage{amsmath}  % for \eqref
\usepackage{amssymb}  % for full AMS mathematics symbology

% Graphics
\usepackage[dvips]{graphics}       	%for embedding graphics files
\usepackage{graphicx}  				%for \includegraphics{file.eps}  % note EPS format
		% (can make a PNG into EPS using png2EPS)

% Set Margins:
%\usepackage{fullpage}  %for not so huge BOOK margins.  Gives std 1" margins all around.
		% my opinion: this is much less pleasantly readable.
% Manually setting margins
%	\addtolength{\oddsidemargin}{-.375in}
%	\addtolength{\evensidemargin}{-.375in}
%	\addtolength{\textwidth}{.75in}
%	\addtolength{\topmargin}{-.5in}
\addtolength{\textheight}{0.0in}  %reduces bottom margin -- more per page, but still good side margins

% Code Listings and formatted dynamically included program listings
\usepackage{verbatim}  	%for formatting dynamically included program listings
						% e.g. {\footnotesize\verbatiminput{c:/tmp/file.tex}} 

% Defined Code Listing Environment
% for creating environments, see http://theoval.cmp.uea.ac.uk/~nlct/latex/novices/newenvironment.html
\newenvironment{code}[1]  % takes pathfilenametocode as argument
{
{\footnotesize \verbatiminput{#1}}% begin code
{}
}    % end code

\newenvironment{proof}%[1] % takes Proof Title as (optional) argument
{
{Proof\\ 
%\begin{proof}{#1}\end{proof} 
\footnotesize}
{}
}

% MACRO DEFINITIONS %
\newtheorem{theorem}{Theorem} %[section]
\newtheorem{claim}[theorem]{Claim}
\newtheorem{proposition}[theorem]{Proposition}
\newtheorem{lemma}[theorem]{Lemma}
\newtheorem{corollary}[theorem]{Corollary}
\newtheorem{fact}[theorem]{Fact}
\newtheorem{definition}[theorem]{Definition}
\newtheorem{question}[theorem]{Question}
\newtheorem{keypoint}[theorem]{Key Point}
\newtheorem{keyconcept}[theorem]{Key Concept}
\newtheorem{problem}[theorem]{Problem}
\newtheorem{exercise}[theorem]{Exercise}
\newtheorem{algorithm}[theorem]{Algorithm}
%\newtheorem{proof}[theorem]{Proof}
%Example: 
%	\begin{lemma}\label{theorem_trivial}<add your Lemma Name>\end{lemma}
%	<add your lemma text>
%   Later, to refer to the label: Theorem \ref{theorem_trivial}.  Note run LaTeX twice to get refs correct.

% NUMBERING CONVENTIONS %
%\numberwithin{equation}{section}  % or {subsection}  % gives equation numbering by section of text
%\numberwithin{theorem}{section}   % numbering choice for theorems, etc.

% MACRO COMMANDS %already defined: \lim, \sin, \min, \arctan, etc.
\newcommand{\qed}{\ensuremath{_\square}} %for end of proof symbol
\newcommand{\lcm}{\ensuremath{\mbox{lcm} }} % LCM -- because \gcd is defined
\newcommand{\imply}{\Longrightarrow} % for =>
\newcommand{\limply}{\Longleftarrow} % for =>
\newcommand{\biimply}{\Leftrightarrow} % for <=>
\newcommand{\goesto}{\rightarrow} % for limit ->
\newcommand{\fuzzy}[1]{\ensuremath{ \underset{\sim}{#1} }} % usage: \fuzzy{A} for A under-tilde

%Macro for: \begin{equation} formula_text \end{equation}
\newcommand{\be}{\begin{equation}}
\newcommand{\ee}{\end{equation}}
%Labeling equations:  \be ... \label{labelname} \ee
%Later, to refer to the label: \ref{labelname} or \eqref{labelname}
